\chapter{Introduction}

This chapter intends to provide the reader with an overview of the current state of diabetes healthcare as well as explain the outline, aim and limitation of this thesis project.

\section{Project Introduction}

In every country the disease burden related to diabetes is already high, and it is steadily increasing \parencite{Forouhi2014}.
In 2017 the estimated prevalence of diabetes was 451 million people globally and approximately 5 million deaths were attributed to diabetes \parencite{Cho2018}.
Aside from reduced life expectancy, diabetes increase the risk of multiple other conditions such as heart disease, stroke and peripheral vascular diseases \parencite{Forouhi2014}.
With a projected prevalence of 693 million diabetes patients in 2045 and given the seriousness of diabetes, proper medical care for patients are of utmost importance \parencite{Cho2018}.

Diabetes is a group of metabolic diseases that is characterized by hyperglycemia resulting from defects in insulin secretion, insulin action, or both \parencite{ADA2010}.
Insulin is necessary to maintain normal blood glucose (BG) levels by facilitating cellular glucose uptake and regulating carbohydrate metabolism \parencite{Wilcox2005}.
The vast majority of cases of diabetes fall into two broad categories, type 1 diabetes (T1D) or type 2.
T1D is caused by an absolute deficiency of insulin secretion, thus
patients need to induce exogenous insulin on a regular basis to maintain balanced BG levels \parencite{ADA2010}.

Maintaining balanced BG levels is an every day challenge of T1D patients.
Treatment guidelines put heavy emphasis on self management activities that benefits a balanced regulation.
This include activities such as eating patterns, exercise and carbohydrate consumption.
Patients favorably consults a clinician regularly to get consultation regarding evalation and making plans for the self management process \parencite{Cooke2013}.


Continuous glucose monitoring (CGM) sensors are wearable devices that measures the blood glucose frequently (usually every 1-5 minutes).
The data CGM sensors provide have revolutionized the ability for clinicians to review a patients data and deliver care driven by consistent data \parencite{Facchinetti2016}.
The data provide effictive insigt in immediate processes such as accuracy of a specific insulin dosing, the BG concentration peak of a certain meal etc.
However, there is a lack of research in long term medical advice derived from CGM data.

Generating insigt in long term outcomes of a T1D historical data could provide benefits of both patients and clinicians.
Patints could through very hand on interaction gain insight in which activities are...

\section{Project Aim}

The general idea is to investigate the potential of leveraging CGM data in personalized T1D care.
Currently the relevant implementations immediate actions such as alerts of high or low BG.

\textit{Extend this section...}
