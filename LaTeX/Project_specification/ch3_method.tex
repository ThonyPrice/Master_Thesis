\chapter{Method}

\section{Data}

The data is collected by Steady Health during a 4,5 month pilot study including 20 patients.
All patients are at least 18 years old and have been injecting insulin at least 1 year prior to the study.

Each patient are equipped with a CGM sensor for the entirety of the study.
For each patient, the CGM collected BG data at intervals of 1-5 minutes.
Patients will also log other events such as meals and exercise making those values accessible at some of the times steps.

\medskip
\begin{table}[ht!]
\begin{center}
  \begin{tabular}{lll}
  \textbf{Notation} &
  \textbf{Field} &
  \textbf{Format} \\
  \hline
  $P$ &
  Patient &
  Integer \\
  $t$ &
  Time &
  Date \\
  $t_{BG}$ &
  Glucose value at $t$ &
  Float \\
  $t_{ACT}$ &
  Physical activity at $t$ &
  Integer \\
  $t_{INS}$ &
  Injected insulin at $t$ &
  Float \\
  $t_{IMG}$ &
  Food image at $t$ &
  Float \\
  $t_{EVENT}$ &
  Manually reported by patient at $t$ &
  Text \\
  \hline
  \end{tabular}
  \caption[]
  {\small The data for each patient include continuous measurements at time steps $t$ of intervals between 1-5 minutes. Each measurement at $t$ \textit{always} include BG value and \textit{may} contain other field presented in the table.}
  \label{table:data_description}
\end{center}
\end{table}


\section{Implementation}

The objective of the proposed system is to analyse the data for a closed time interval, identify meaningful events and classify them accordingly.
The analasys is not performed in real time but is considered a batch problem.
The proposed steps of implementation can be overviewed in figure [FIGURE REF].
Each step is more detailedly described in its subsequent section below.

\subsection{Online Denoising}

The data gathered with CGM may include sensor noise.
Noise can trigger false positives in event detection and small fluctuations hides the true underlying derivatives of the curve.
As proposed by [REF] applying a [X] filter to produce a filtered signal.
[Math behind gilet goes here].
The result of aplying such filter can be seen in figure [FIGURE REF].

\subsection{Qualitative Representation}

To identify events in the denoised CGM data feature extraction is used.
Feature extraction can be acheived by either a qualitative or quantitative method.
The qualitative method offer benefits such as more transparent reasoning and ability to provide explanations for for solutions provided \parencite{Ven2003}.

Triangular representation...

Membership degree...

\subsection{Intervention Analysis}

Detect event with impact on trend average.

\section{Evaluation}

Training/test...

Padova...

Manual labeling...
