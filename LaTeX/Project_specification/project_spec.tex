\documentclass{kththesis}

\usepackage[linesnumbered,ruled]{algorithm2e}
\usepackage{amsmath}
\usepackage{booktabs}
%\usepackage{biblatex}
\usepackage{chngpage}
\usepackage{csquotes}
\usepackage{float}
\usepackage[final]{pdfpages}
\usepackage{graphicx}
\usepackage{hyperref}
\usepackage{longtable}
\usepackage{mwe}
\usepackage{multirow}
\usepackage{numprint}
\usepackage[parfill]{parskip}
\usepackage{placeins}
\usepackage{subcaption}
\usepackage[backend=biber]{biblatex}
\addbibresource{references.bib}

\title{%
    Master Thesis - \\
    Project Plan and Specification \\
    \vspace{.5em}
    \large Meal Detection and Carbohydrate Estimation from Continuous Glucose Monitoring data \\
}

\alttitle{Detta är den svenska översättningen av titeln}
\author{Thony Price}
\email{ThonyP@kth.se}
\supervisor{Pawel Herman}
\examiner{-}
\programme{Master in Computer Science}
\school{School of Electrical Engineering and Computer Science}
\date{\today}

\begin{document}
\frontmatter
\titlepage

\begin{abstract}
  About this document...

  Diabetes is on the rise...
  Proper care comes from good data...
  Diet is of major concern...
  CGM allows easy data collection...
  Meal detection is...
  Carbohydrate estimation...
\end{abstract}

\tableofcontents
\mainmatter

\chapter{Introduction}

In every country the disease burden related to diabetes is high and on the rise \parencite{Forouhi2014}.
In 2017 the estimated prevalence of diabetes was 451 million people globally and approximately 5 million deaths were attributed to diabetes \parencite{Cho2018}.
Besides mortality, diabetes increase the risk of heart disease, stroke, peripheral vascular diseases along with overall reduced life expectancy \parencite{Forouhi2014}.
With these factors in mind, along with a projected prevalence of 693 million diabetes patients in 2045, proper medical care for diabetes are of utmost importance \parencite{Cho2018}.

Diabetes is a group of metabolic diseases characterized by hyperglycemia resulting from defects in insulin secretion, insulin action, or both \parencite{ADA2010}.
Insulin is necessary to maintain normal blood glucose levels by facilitating cellular glucose uptake and regulating carbohydrate metabolism \parencite{Wilcox2005}.
The vast majority of cases of diabetes fall into two broad categories, type 1 and type 2.
Type 1 is caused by an absolute deficiency of insulin secretion, thus
patients need to induce exogenous insulin on a regular basis to balance maintain balanced blood glucose levels \parencite{ADA2010}.

Today no cure of diabetes exists but self management by patients can improve the long term outcome.
Self management actions include activities such as eating patterns, exercise, alcohol and carbohydrate consumption \parencite{S38}.
Patients.2

[Brief section on A1C value as indicator of patient's situation].
Medical care guidelines put emphasis on diet and lifestyle factors to improve the A1C value \parencite{ADA2018}.
Commonly a specialist works with weekly measurement to asses the current development of a patient and derives recommendations from that. [Rewrite this paragraph...]

Continuous glucose monitors (CGMs) are wearable devices that measures the blood glucose frequently. The usability offer a larger dataset to evaluate a patient's fluctuations of glucose values. The

CGM's potential to improve treatment...

\section{Research Question}

Text...

\section{Scope}

\subsection{Purpose}

From a study performed by Steady health unique CGM data will be availiable...

\subsection{Limitations}

The data is collected from various devices...

In vito tests of recommendations is not included in the study...

Therefore evaluation is performed on PODOVA..?

\chapter{Background}

\section{Glossary}

Include here...

\section{Diabetes}

\subsection{Definition}

Text...

\section{Tool 1}
Text...

\section{Tool 2}
Text...

\section{Tool 3}
Text...

\section{Previous Work}

\chapter{Methods}

\section{Data}
Text...

\section{Implementation}
Text...

\section{Evaluation}
Text...

% Print the bibliography (and make it appear in the table of contents)
\printbibliography
%\printbibliography[heading=bibintoc]

% \appendix
% \chapter{Unnecessary Appended Material}

\end{document}
