\documentclass{kththesis}

\usepackage[linesnumbered,ruled]{algorithm2e}
\usepackage{amsmath}
\usepackage{booktabs}
\usepackage[backend=biber]{biblatex}
\usepackage{chngpage}
\usepackage{csquotes}
\usepackage{float}
\usepackage[final]{pdfpages}
\usepackage{graphicx}
\usepackage{hyperref}
\usepackage{longtable}
\usepackage{mwe}
\usepackage{multirow}
\usepackage{numprint}
\usepackage[parfill]{parskip}
\usepackage{placeins}
\usepackage{subcaption}
\addbibresource{references.bib}

% --- TITLE CONTENT ---
\title{%
    Master Thesis - \\
    Project Specification \\
    \vspace{.5em}
    \large HbA1c Impact Estimation of Detected Events from Continuous Glucose Monitoring Data\\
}

\alttitle{Estimerad HbA1c Inverkan Av Detekterade Event ur Kontinuerlig Glukos Matardata}
\author{Thony Price}
\email{ThonyP@kth.se}
\supervisor{Pawel Herman}
\examiner{-}
\programme{Master in Computer Science}
\school{School of Electrical Engineering and Computer Science}
\date{\today}

\begin{document}
\frontmatter
\titlepage

% --- ABSTRACT ---
\begin{abstract}
  Diabetes affect millions of people today and health care emphasize on self management by the patients.
  Continuous glucose monitoring sensors have enabled close monitoring of the current and historical blood glucose concentration for both patients and clinicians.
  This data may also include valuable insight about patterns that have an impact on the long term progression of blood glucose development in patients.

  This thesis aims to investigate a new approach to detect events in blood glucose data and estimate their impact on the long term blood glucose concentration progression.
  The method includes filtering the signal, derive a qualitative representation which is fed to a hidden Markov model.
  The model detects events which by intervention analysis is labeled with an estimated impact.
\end{abstract}

\tableofcontents
\mainmatter

\chapter{Introduction}

This chapter intends to provide the reader with an overview of the current state of diabetes healthcare as well as explain the outline, aim and limitation of this thesis project.

\section{Project Introduction}

In every country the disease burden related to diabetes is already high, and it is steadily increasing \parencite{Forouhi2014}.
In 2017 the estimated prevalence of diabetes was 451 million people globally and approximately 5 million deaths were attributed to diabetes \parencite{Cho2018}.
Aside from reduced life expectancy, diabetes increase the risk of multiple other conditions such as heart disease, stroke and peripheral vascular diseases \parencite{Forouhi2014}.
With a projected prevalence of 693 million diabetes patients in 2045 and given the seriousness of diabetes, proper medical care for patients are of utmost importance \parencite{Cho2018}.

Diabetes is a group of metabolic diseases that is characterized by hyperglycemia resulting from defects in insulin secretion, insulin action, or both \parencite{ADA2010}.
Insulin is necessary to maintain normal blood glucose (BG) levels by facilitating cellular glucose uptake and regulating carbohydrate metabolism \parencite{Wilcox2005}.
The vast majority of cases of diabetes fall into two broad categories, type 1 diabetes (T1D) or type 2.
T1D is caused by an absolute deficiency of insulin secretion, thus
patients need to induce exogenous insulin on a regular basis to maintain balanced BG levels \parencite{ADA2010}.

Maintaining balanced BG levels is an every day challenge of T1D patients.
Treatment guidelines put heavy emphasis on self management activities that benefits a balanced regulation.
This include activities such as eating patterns, exercise and carbohydrate consumption.
Patients favorably consults a clinician regularly to get consultation regarding evalation and making plans for the self management process \parencite{Cooke2013}.


Continuous glucose monitoring (CGM) sensors are wearable devices that measures the blood glucose frequently (usually every 1-5 minutes).
The data CGM sensors provide have revolutionized the ability for clinicians to review a patients data and deliver care driven by consistent data \parencite{Facchinetti2016}.
Care on CGM data have proven effective in lowering patients long term average BG concentration (A1C) \parencite{dungan2018monitoring}.
The data provide effictive insigt in immediate processes such as accuracy of a specific insulin dosing, the BG concentration peak of a certain meal etc.
However, there is a lack of research in deriving insights for long term medical advice autonomous from CGM data.

\section{Project Aim}

Generating insigt in long term outcomes of a T1D patient's historical data could provide benefits to both patients and clinicians.
Patients could receive feedback automated feedback of current habits and its potential impact on long term A1C.
Clinicians benefit from automated insights in a patient's be reducing the risk of missing patterns.
Autonomous insights can be reviewd by clinicians as second opinion to form a more objective care plan too.

\subsection{Objective}

The objective of this thesis is to investigate how an automonous system, which fulfills the cartain criteria below, can be implemented and in which configuration it perfoms optimally. The system should:
\begin{itemize}
  \item Detect events in from a batch CGM time series data (such as meal, exercise or sleep).
  \item Classify the event of the event.
  \item Estimate intervention caused by an event (what impact did the event have on the continued time series).
\end{itemize}

\subsection{Research Question}

Can techniques and algorithms X, Y, Z provide a system complying with the specifications defined in Objective?

\chapter{Background}

This chapter sets out define diabetes and relevant aspects of the enough to grasp the content in latter chapters. It further describes theory of the techniques in the suggested method and reviews related work that should assist putting this thesis into context of the current field.

\section{Diabetes}

\subsection{Definition}
Text...

\subsection{Self Management}
Text...

\section{Wavelet Filter}
Text...

\section{Qualitative Representation}
Text...

\section{Event Detection Tool}
Text...

\section{Intervention Analysis}
Text...

\section{Previous Work}
Text...

\chapter{Method}

\section{Data}

The data is collected during a 4,5 month pilot study which includes 20 patients.
All patients are at least 18 years old and have been injecting insulin at least 1 year prior to the study.
Each patient are equipped with a CGM sensor for the entirety of the study.
For each patient, the CGM sensor measures the blood glucose concentration (BGC) at intervals of 1-5 minutes.
The (time, BGC) tuples constitutes the structure of each patient's data set.
Additionally, other events such as meals and exercise are logged manually by patients.
A summary of collected data points are presented in Table \ref{table:data_description}.

\medskip
\begin{table}[ht!]
\begin{center}
  \begin{tabular}{lll}
  \textbf{Notation} &
  \textbf{Field} &
  \textbf{Format} \\
  \hline
  $P$ &
  Patient &
  Integer \\
  $t$ &
  Time &
  Date \\
  $t_{BG}$ &
  Glucose value at $t$ &
  Float \\
  $t_{ACT}$ &
  Physical activity at $t$ &
  Integer \\
  $t_{INS}$ &
  Injected insulin at $t$ &
  Float \\
  $t_{IMG}$ &
  Food image at $t$ &
  Float \\
  $t_{EVENT}$ &
  Manually reported by patient at $t$ &
  Text \\
  \hline
  \end{tabular}
  \caption[]
  {\small The data for each patient include continuous measurements at time steps $t$ of intervals between 1-5 minutes. Each measurement at $t$ \textit{always} include BG value and \textit{may} contain other field presented in the table.}
  \label{table:data_description}
\end{center}
\end{table}


\section{Implementation}

The objective of the proposed system is to analyze the data for a closed time interval, identify events and classify them accordingly with respect to their influence over future measurements.
The analysis is not performed in real time, all data is available immediately to the algorithms.
The proposed steps of implementation can be overviewed in Figure \ref{fig:flowchart}.
Each step is described in more detail in its corresponding section below.

\begin{figure}[ht!]
  \centering
  \includegraphics[width=0.8\textwidth]{images/flowchart.jpg}
  \caption[]
  {\small Schematics of implementation.}
  \label{fig:flowchart}
\end{figure}


\subsection{Wavelet Filter}

Studies have shown data from CGM sensors is subject to distortion.
This is caused by diffusion processes and by time-varying systematic under/overestimations due to calibrations and sensor drifts \parencite{facchinetti2014modeling}.
Noise can trigger false positives in event detection because abrupt fluctuations overrides the true underlying derivatives of the curve \parencite{Facchinetti2016}.
Wavelet filters have been used repeatedly with CGM data and proved successful in reducing noise while retaining events such as spikes \parencite{Mag2016}, \parencite{Facchinetti2016}, \parencite{samadi2017}.
Figure \ref{fig:wavelet_example} displays an example of wavelet filtering applied to CGM data.

\begin{figure}[ht!]
  \centering
  \includegraphics[width=0.6\textwidth]{images/wavelet_example.jpg}
  \caption[]
  {\small Wavelet filter applied on CGM data. Vertical axis represents glocose concentration [mg/dl]. Image courtesy of \textcite{samadi2017}.}
  \label{fig:wavelet_example}
\end{figure}


\subsection{Qualitative Representation}

To identify events in the de-noised CGM data, feature extraction is used.
Feature extraction can be achieved by either a qualitative or quantitative method.
The qualitative method offer benefits such as more transparent reasoning and ability to provide explanations for for the solutions it provides \parencite{Ven2003}.

In qualitative representation by triangular shapes, a CGM data segment can take seven shape variables.
Figure \ref{fig:shapes} shows the different shapes.
Each is a unique combination of the first and second order derivatives on the curve of the current segment.
The derivates can be read from segment of adjacent points, allowing the CGM data series to be presented as a sequence of shapes describing fluctuations in BGC.

\begin{figure}[ht!]
  \centering
  \includegraphics[width=0.6\textwidth]{images/shapes.jpg}
  \caption[]
  {\small Scheme of the qualitative variables A-G. Image courtesy of \textcite{samadi2017}.}
  \label{fig:shapes}
\end{figure}


\subsection{Event Detection}

With the qualitative representation, event detection can be performed by analyzing the sequence of shapes.
In figure \ref{fig:curve2shape} an event could be triggered by identifying a continuously accelerating increase (for example the four sequential D's from time-step 12).

\begin{figure}[ht!]
  \centering
  \includegraphics[width=0.8\textwidth]{images/curve2shape.jpg}
  \caption[]
  {\small Shape sequence representation if CGM curve. Image courtesy of \textcite{samadi2017}.}
  \label{fig:curve2shape}
\end{figure}


Given a sequence of observations (i.e. shapes) with some underlying state (i.e. eating, sleeping, exercising) a hidden Markov model can provide an estimate of what state a current time step most probably corresponds to. The Markov approach includes two steps:

\textbf{The evaluation problem}: Calculating probability distribution by training a model on sequences with labeled states.

\textbf{The decoding problem}: Determining the hidden states by observing a unlabelled sequence and finding the optimal state sequence associated with the given observation sequence \parencite{rabiner1989}.

\subsection{Intervention Analysis}

Intervention analysis provides a tool to asses how much a given event has changes the series (if at all) \parencite{box2015time}.
The analysis is able to detect 4 patterns:

\begin{enumerate}
  \item Permanent constant change to the mean level.
  \item Brief constant change to the mean level.
  \item Gradual increase or decrease to a new mean level.
  \item Initial change followed by gradual return to previous mean level.
\end{enumerate}

The mean level in our case is the mean BGC.
Intervention analysis compare the mean average before the event and its progression afterwards.
The characteristics of the progression curve defines which of the four patterns an event has triggered.
Because changes in mean BG concentration are subtle and changes takes place over a longer timespan, an alternative to the approach is suggested.

An event by itself may not cause a observable change in the mean average but combined events might.
The intervention analysis is also evaluated on sets of adjacent events.
The output can hint on when the aggregated data implies a positive development of the mean.
In other words, the analysis can suggest when combined activities by a patient results in a change of his or her long term HbA1c value.

\section{Evaluation}

Training/test...

Padova...

Manual labeling...


\printbibliography[heading=bibintoc]
% \appendix
% \chapter{Unnecessary Appended Material}

\end{document}
